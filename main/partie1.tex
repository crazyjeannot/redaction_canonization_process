\part{Préliminaires}

\chapter{Les modalités du canon littéraire}
%wrong title 

\section{\textit{Canon} et \textit{classiques} de la littérature}

Nous utiliserons dans ce mémoire la notion de \textit{canon littéraire} dans le sens où ce dernier est composé d'ouvrages \textit{classiques}, ou d'auteurs \textit{classiques}. 

Le mot canon vient du grec \enquote{\foreignlanguage{greek}{Κανων}} qui signifie \enquote{roseau} ou \enquote{tige}, littéralement \enquote{droit comme une tige de roseau} \footnote{William Marx,  cours au collège de France, \url{https://www.college-de-france.fr/site/william-marx/course-2021-04-13-14h30.htm}} utilisé comme un instrument de mesure. Avec le temps le terme prend le sens de \enquote{règle} ou de \enquote{loi}, et c'est avec cette signification qu'il arrive dans les langages européens modernes. C'est au quatrième siècle après JC\footnote{John Guillory, \enquote{Canon} in \cite{lentricchia_critical_2012},  p.233-245}, avec le canon biblique, que le mot prend le sens actuellement utilisé dans les études littéraires. L'Église instaure une liste de textes formant les Saintes Écritures. Le canon avait alors une valeur d'autorité qui permettait à l'Église de contrôler un récit commun pour la postérité.

%le canon comme bibliothèque mentale collective. cf William Marx

La notion a été introduite dans les études littéraires pour désigner les syllabus des universités et les textes qui y étaient étudiés. Selon certains chercheurs, le canon est nécessaire à l'enseignement de la littérature : \enquote{L'étude institutionnelle de la littérature est inconcevable sans un canon. Sans canon, sans corpus ou syllabus de textes exemplaires, il ne peut y avoir de communauté interprétative.}\footnote{\cite{felperin_beyond_1985}, les traductions des citations ont été réalisés par mes soins}. Le canon est ainsi l'ensemble des textes sur lesquels se fonde la discipline et la recherche en littérature. Pour Pascale Casanova, le canon \enquote{incarne la légitimité littéraire elle même, c'est à dire ce qui est reconnu comme \textit{la littérature}, et qui servira d'unité de mesure spécifique}\footcites{casanova_republique_2008}.  

W. Harris distingue différentes fonctions du canon littéraire\footcites{harris_canonicity_1991} : Il permet de \enquote{fournir des modèles, de transmettre l'héritage de la pensée, par la provision de savoirs culturels nécessaires pour interpréter les textes du passé}. Le canon crée des cadres de référence communs pour faire société par identification à un ensemble de mythes et de textes sacrés. Un des problèmes soulevés par ce genre d'approche du canon est que si l'on se penche sur les listes des grands auteurs qui les composent, on trouvera très peu de femmes, et encore moins d'auteurs non blancs, ou de classe sociale défavorisée\footcites{guillory_cultural_1998}. Les processus de formation du canon littéraire sont souvent exclusifs, et il faudrait déconstruire les préjugés sur lesquels se fonde le canon.

Les polémiques sur le canon littéraire ont été particulièrement virulentes dans les campus universitaires étasuniens à partir des années 1980. La synthèse d'Eric Fassin\footcites{fassin_chaire_1993} est à cet égard fort instructive. Deux camps s'affrontent, avec d'un côté les défenseurs du canon (compris comme la quintessence de la tradition littéraire occidentale\footcites{bloom_western_1994}) au nom des valeurs universelles qu'il incarne. De l'autre côté, ses détracteurs, qui relèvent les conditions sociologiques, géographiques, idéologiques et culturelles de la production des critères de définition et de sélection du canon littéraire\footcites{harder_deconstruire_2013}. 

En France, la notion de canon n'est pas si répandue et doit son existence aux polémiques venant d'outre-Atlantique. C'est le terme \textit{classique} qui est utilisé, et qui n'est d'ailleurs pas dédié uniquement à l'espace littéraire. Le nom \textit{les Classiques} désigne aujourd'hui les grands écrivains de la littérature française, mais cela n'a pas toujours été le cas.

Le mot \textit{classique} vient du vocabulaire de la richesse et de la propriété. Du latin \textit{classicus}, signifiant \enquote{citoyen de la première classe}\footnote{Trésor de la langue française informatisé ; https://atilf.fr/ressources/tlfi}. L'adjectif \textit{classique} prend son sens de jugement esthétique au XVII\ieme siècle, où il signifie ce qui mérite d'être copié ou de servir de modèle. A la fin du XVII\ieme siècle, la signification dérive et prend le sens de ce qui était enseigné en classe. Au cours du XVIII\ieme siècle, le terme désigne les auteurs antiques grecs ou latins. 

Finalement au cours du XIX\ieme siècle, on appelle \textit{classiques}, par opposition aux auteurs romantiques contemporains, les auteurs du temps de Louis XIV et cette époque comme le classicisme français\footcites{viala_quest-ce_1993}. Ainsi, dans son sens le plus restreint, le \textit{classique} désigne les auteurs de théâtre du XVII\ieme siècle, considéré comme le grand siècle de la littérature française puisque cette période littéraire s'est le plus rapproché de la perfection antique.

L'enjeu du terme \textit{classique} se joue au niveau de la réception. Ce n'est pas l'auteur qui choisit d'être classique, c'est l'institution scolaire ou la critique, et ses représentants, qui choisissent et émettent le jugement esthétique et politique de son appartenance au cercle des \textit{classiques}.

Sainte-Beuve, critique et écrivain français conservateur du milieu XIX\ieme siècle nous donne un bon échantillon de la perception des classiques au XIX\ieme siècle :
\begin{displayquote} \enquote{Un vrai classique, comme j'aimerais à l'entendre définir, c'est un auteur qui a enrichi l'esprit humain, [...] qui a découvert quelque vérité morale non équivoque, [...] qui a parle à tous dans un style à lui et qui se trouve aussi celui de tout le monde, dans un style [...] nouveau et antique, aisément contemporain de tous les âges.}\footnote{Sainte-Beuve, \textit{Causeries du lundi}, cité dans\cite{compagnon_sainte-beuve_1995}}
\end{displayquote}
Plusieurs éléments sont intéressants. La notion de style - que nous discuterons plus tard dans cette partie - apparaît centrale pour définir le classique. Sainte-Beuve ne nous aide pas en présentant avec deux oxymores le style classique, qui serait à la fois personnel et commun, \enquote{nouveau et antique}. Il y a une forme de sacralisation du classique, qui est selon Sainte-Beuve un idéal littéraire que peu d'écrivain ont atteint. Sainte-Beuve essentialise le classique en le rapprochant de la notion de vérité, et justifie ainsi la hiérarchie entre les classiques et les autres.

% Ce jugement de l'esthétique supérieur du style classique est une forme de sacralisation du classique /qui met à distance la littérature contemporaine de la consécration du classique./

\section{Le canon et l'enseignement}

La supériorité esthétique mise en avant par Sainte-Beuve n'est pas suffisante pour expliquer les différents processus de canonisation des auteurs. La relativité du jugement esthétique des critiques ne permet pas d'expliquer la stabilité du canon littéraire. 

C'est dans l'enseignement de la littérature que l'on peut constater les dynamiques de construction et de modification du canon littéraire français. L'institution scolaire est un des lieux majeurs où le canon s'élabore. En effet, elle produit des panthéons d'auteurs et de textes, souvent sous la forme de morceaux choisis\footcites{jey_litterature_1998} pour éduquer des générations d'écoliers. Dès la première liste nationale de 1803, le canon des auteurs français est constitué. Il est repris ensuite pour être diffusé grâce au système scolaire centralisé.

L’institution scolaire construit une représentation de la littérature qui lui est propre. Elle en détermine le bon usage, avec des découpages chronologiques (périodisations, écoles, générations), une utilisation de catégories (romantisme, naturalisme, surréalisme), et enfin l'élaboration d’un canon par une sélection d’auteurs. Ces classiques sont le fruit d’une sélection particulièrement étroite. Ils sont présentés comme des modèles, véhiculant une norme esthétique particulière. C'est cette norme qui nous intéresse et que nous voulons repérer quantitativement.

Le canon littéraire construit par l'institution scolaire devient ainsi matrice de formes spécifiques en tant qu'il est le représentant de ce qui est considéré comme de la \textit{bonne} littérature. Pourtant le canon n'est pas monolithique et s'ouvre au cours du temps. Les garants du bon ordre littéraire (l'institution scolaire et, dans une moindre mesure, la critique) nourrissent et ouvrent le canon sous la pression et l'usure du temps aux oeuvres qui semblent le plus en accord avec une certaine idée de la littérature. 

\section{Un canon politique}

Le canon littéraire porte une dimension politique, tant dans sa construction que dans ce qu'il représente en tant que tel. Selon Alain Viala, les auteurs retenus dans le canon \enquote{remplissent une fonction d’identification culturelle}\footcites{viala_quest-ce_1993}, c'est à dire qu'ils représentent le socle commun de la construction culturelle d'une nation. Dans le contexte français, l'évolution du canon littéraire est le témoin de la légitimation et de l'affirmation d'une langue et d'une culture française par rapport au latin. La première liste institutionnelle d'auteurs de 1803 associe un auteur grec ou latin à un auteur français du XVII\ieme siècle. Avec la structuration et la centralisation du système scolaire sous la troisième république, le canon se cristallise et devient un objet politique\footcites{compagnon_troisieme_1983}. La littérature dans l'enseignement, associée à un canon littéraire, se trouve investie d'un rôle d'éducation des masses et de diffusion de valeurs nationales. On peut trouver un exemple dans le canon des agrégations, qui est \enquote{ramassé autour de ce qui semble incarner les valeurs nationales}\footcites{jey_canon_2014}. Différentes réformes de l'enseignement scolaire ont façonné le canon et la façon d'enseigner la littérature au cours du temps.

Après cette première ouverture du canon littéraire, ce dernier a longtemps été la chasse gardée des auteurs du XVII\ieme siècle, et très peu de genres littéraires étaient représentés. Seul le théâtre et la poésie étaient les genres autorisés, puisqu'ils étaient les garants d'un classicisme formel, modèle d'ordre, de clarté, le tout dans un équilibre et une harmonie parfaits.

A la fin du XIX\ieme siècle, la société évoluant vers une laïcisation des moeurs sociales, l'enseignement de la littérature fait face à des difficultés. Gustave Lanson, célèbre historien et critique littéraire, le constate : 
\begin{displayquote} \enquote{C'est une absurdité de n'employer qu'une littérature monarchique et chrétienne à l'éducation d'une démocratie qui n'admet point de religion d'Etat.}\footnote{Gustave Lanson, «L'étude des auteurs français», Revue universitaire, 1894}
\end{displayquote}

Pour G. Lanson, la République doit se doter de nouveaux \enquote{textes sacrés}\footcites{compagnon_demon_1998}, et le canon littéraire doit s'ouvrir aux écrivains et aux genres littéraires contemporains. Les missions de la littérature sont alors de moraliser les moeurs sociales et de fabriquer un consensus national. Le roman entre ainsi au panthéon de la littérature, et avec lui la plupart des auteurs romanesques considérés encore aujourd'hui comme des classiques de la littérature française. 

%\enquote{C'est un genre littéraire aussi jeune que l'idée de nation, le roman, qui va à la fois servir de modèle narratif pour les premières élaborations savantes d'histoires nationales et de formidables vecteur de diffusion d'une vision nouvelle du passé}\footcites{thiesse_fabrique_2019}.

Lors de la poussée des nationalismes du début du XX\ieme siècle, la littérature doit affirmer l'unité voire la supériorité culturelle de la France. Ces processus ont été étudiés par Anne-Marie Thiesse dans son livre \textit{La fabrique de l'écrivain national : entre littérature et politique}. 
\begin{displayquote}\enquote{La nationalisation de l'état passe au premier chef par un intensif travail d'éducation de masses, visant à inculquer dans l'ensemble de la population le sentiment d'appartenance commune}\footcites{thiesse_fabrique_2019}. \end{displayquote}

Ainsi, le canon littéraire incarne les fondements du récit national. Il peut être vu comme la construction politique de la culture nationale. C'est d'ailleurs un paradoxe que souligne Pascale Casanova dans son livre \textit{La république mondiale des lettres}\footcites{casanova_republique_2008}. Les cultures européennes se fondent sur des canons nationaux, tout en revendiquant les valeurs transnationales et universelles de ces derniers, qui prendraient racine dans un passé lointain des antiques grecs et latins.


\section{Le style et les classiques}
La construction du canon littéraire est éminemment politique, sa mise en place et sa diffusion relevant de politiques culturelles de mise en avant d'une littérature spécifique, censée porter les valeurs des régimes successifs de l'histoire de France. Les architectes du canon ont fondé leur sélection sur des critères linguistiques. Ils ont sélectionné ce qui, selon eux, représentait le meilleur usage de la langue française. Le classique est un modèle de style, et c'est pour cela qu'il est enseigné. Si nous pouvons questionner cette sélection en tant qu'elle lui donne un aspect arbitraire, le jugement esthétique étant par nature relatif, nous ne pouvons pas l'écarter complètement.

On peut définir le style comme \enquote{l'ensemble des traits expressifs qui dénotent l'auteur dans un écrit}\footnote{ Trésor de la Langue Française informatisé, \url{http://stella.atilf.fr/}}. En littérature, on assimile souvent le \textit{style} à l'expression de la singularité de l'auteur, en tant qu'il s'écarte de la \textit{norme}. Pour Georges Molinié, grand stylisticien, \enquote{Il n'y a de style que dans la mesure où des régularités dans le choix permettent de caractériser une écriture}\footcites{molinie_quest-ce_1994}. Le style est un ornement formel, qui est propre à chaque auteur. 



Toujours selon Georges Molinié, les qualités stylistiques des classiques pourraient expliquer leur filtration temporelle : 

\begin{displayquote}
\enquote{Dans la masse des romans et des pièces de théâtre, très peu, [...] sont encore lus de nos jours : l'invention et la gestion des situations dramatiques sont bien moins en cause que la qualité, la force, l’éclat d’écriture. C’est d’abord cette différence stylistique qui a séparé Pradon de Racine ; c’est la phrase de Madame de La Fayette qui a durablement séduit, parmi d’autres écritures effectives, indépendamment des atavismes ou des ruptures de la composition et de l’invraisemblable des situations. [...] Il est difficile de nier cette primauté du style.}\footcites{molinie_style_1996}
\end{displayquote}

La pérennité du canon littéraire peut s'expliquer par la \enquote{primauté du style} présente dans les \textit{classiques} de la littérature. La langue littéraire, que l'on pourrait définir comme la langue des écrivains, se caractérise par son style, en opposition à la langue de tous les jours, qui manque de style. 

Il faudrait revenir sur notre définition du style, et à la tension qui existe entre norme et style. On a vu que le style était traditionnellement vu comme l'expression d'une singularité, c'est à dire d'un idiolecte propre à un auteur. Le style peut aussi être vu comme une généralité, une langue littéraire, c'est à dire un sociolecte utilisé par les écrivains. Selon la formule de Gilles Philippe, la littérature, au sens du canon littéraire, est un \enquote{conservatoire consacrant la norme et un laboratoire célébrant la liberté du locuteur\footcites{philippe_langue_2009}}. Le style est indissociable de ces deux aspects, mais le premier a longtemps été minimisé, au profit du second, par les études stylistiques. 
%p30

Le style présent dans le canon littéraire serait \enquote{une norme étalon à partir de laquelle l'on attribue ou l'on refuse une valeur esthétique à une production langagière}\footcites{philippe_pourquoi_2021}. Il y aurait une norme stylistique haute dans le canon littéraire; c'est cette norme que nous voulons détecter quantitativement. Les méthodes quantitatives semblent les mieux adaptées pour envisager l'ensemble des caractéristiques qui forment l'esthétique de la langue littéraire, ou plus simplement, l'esthétique canonique.
%p63

Le style est une notion que les études quantitatives ont abordé. On peut notamment présenter l'article de Herrmann et al, \enquote{Revisiting style, a key concept in literary studies}. Ces chercheurs ont proposé une définition du style compatible avec les méthodes quantitatives : \enquote{Le style est une propriété du texte constitué par un ensemble de caractéristiques formelles, pouvant être observées quantitativement ou qualitativement}\footcites{BerenikeHerrmann2015RevisitingSA}. La puissance de calcul et l'étude de grand corpus dans le cadre des études littéraires computationnelles peut aider à désingulariser la notion de style. Nous proposons, dans ce mémoire, d'étudier le style comme un phénomène collectif, qui caractériserait l'esthétique spécifique du canon littéraire.

Dans le prochain chapitre, nous verrons comment définir l'approche des études littéraires computationnelles pour appréhender des concepts littéraires tels que celui du style.

%faisceaux - ensemble de traits stylistiques

%\enquote{Pour faire vite, on peut dire que, pendant la première moitié du XXe siècle on a ramené la question de la littérature à celle de la langue littéraire et celle de la langue littéraire à celle de la grammaire, considérée comme charpente des productions langagières. [...] La grammaticalisation de l'ensemble des discours sur la littérature a provoqué l'avènement d'un concept nouveau, [...], celui de "littérarité", qui repose sur l'idée d'une spécificité partielle ou complète du discours littéraire face aux discours non littéraire. En mettant au cœur du débat la question de la grammaire, on passait en effet de la définition humaniste de la littérature comme corpus à une définition formaliste de la littérature comme pratique de la langue}\footcites{philippe_sujet_2002}Gilles Philippe, Le moment grammaticale de la littérature française, 1890-1940, p15

%C'est à partir des années 1850 que semble s'être progressivement imposée en France l'idée selon laquelle la langue des écrivains ne pouvait plus être ramenée à la langue commune. L’idiolecte c’est le style sans la signification, et le style, l’idiolecte en tant qu’il peut faire l’objet d’une interprétation.

%Antoine Albalat : La décomposition du style doit être placée en première ligne dans les études critiques, parce qu'elle est \footnote{\textit{Ouvriers et procédés}, p27-28, cité dans \cite{philippe_reve_2013}, p99}



\chapter{Approches quantitatives}

\section{La mesure dans les études littéraires}

%histoire ancienne -> sacy ? dray ? cours stylométrie ? mais développements récents:

Grâce aux progrès de l'informatique et du traitement du langage naturel, l'utilisation d'approches statistiques et mathématiques dans le domaine des études littéraires a considérablement augmenté ces dernières années. 

La notion de \textit{\enquote{distant reading}}\footnote{Franco Moretti. « Conjectures on World Literature », \textit{New Left Review}, 2000. On notera une synthèse des écrits autour de cette notion dans un ouvrage dédié au \textit{\enquote{distant reading}}.\cite{moretti_distant_2013}.}, définie en introduction n'est finalement qu'une nouvelle approche de description et de production de savoirs empiriques en histoire littéraire. 

Ce nouveau champ de recherches permet d'augmenter la focale en prenant en compte la production littéraire dans un ensemble le plus grand possible. Cela permet de comprendre les contrastes entre des oeuvres tirées de différentes périodes ou de différents contextes sociaux. 

Franco Moretti et le laboratoire littéraire de Stanford\footnote{https://litlab.stanford.edu/} ont contribué à développer cette nouvelle approche des textes littéraires fondée sur la mesure et le calcul de fréquences d'éléments du texte. Leurs pamphlets 1, \enquote{Quantitative Formalism : an Experiment}\footcites{allison_quantitative_2011}, 6, \enquote{"Operationalizing” : or, the function of measurement in modern literary theory}\footcites{moretti_operationalizing_2013} et 12, \enquote{Literature, Measured}\footcites{moretti_literature_2016} conceptualisent le rôle de la mesure pour appréhender des objets culturels. Ces pamphlets fondent une discipline, les études littéraires computationnelles.

Ces dernières modifient l'approche conceptuelle de l'objet d'étude des analyses littéraires traditionnelles. Un roman n'est plus vu comme un objet physique singulier, mais plutôt comme un agrégat de signes caractéristiques et quantifiables, dont l'ensemble ne forme qu'un échantillon parmi un corpus de milliers d'autres ouvrages. Mark-Algee Hewitt, un chercheur du domaine, souligne que \enquote{l'enjeu de cette transformation est donc notre compréhension du roman lui-même : ce qu'il est, les informations qu'il contient et ce qu'il peut faire en tant qu'objet littéraire et critique}\footnote{\cite{algee-hewitt_novel_2018}}. Le roman est ainsi envisagé comme un ensemble de données, dont la signification mathématique peut nous apprendre des éléments critiques sur un auteur, un genre littéraire ou même un espace culturel. 

Cet examen systématique des données contenues dans les textes écrits est conceptualisé par Matthew L. Jockers comme une analyse à l'échelle macroscopique\footcites{jockers_macroanalysis_2013} de la littérature. Cette approche \textit{macro-analytique} permet de comprendre \enquote{le degré auquel la littérature et les auteurs individuels qui la fabriquent répondent ou réagissent aux tendances littéraires et culturelles}. C'est une nouvelle manière de décrire la position de l'écrivain dans le champ littéraire. 

Avec ces nouvelles méthodes, le chiffre et la mesure prennent ainsi une influence particulièrement importante. Ted Underwood explicite leur rôle : \enquote{Les chiffres ne sont pas plus objectifs que les mots. Ils ne sont que des signes qui permettent aux observateurs humains de se débattre avec des questions de degré. Nous avons besoin de chiffres pour comprendre les longues chronologies littéraires}\footcites{underwood_why_2018}. Le chiffre permet de manipuler de grands ensembles de données et de chercher du sens face à la quantité énorme de textes écrits. Le livre d'Andrew Piper - \enquote{Énumérations\footcites{piper_enumerations_2018}} - pose la question des implications de la conception de la littérature comme \textit{quantité}. La multiplicité des possibilités des analyses et la quantité massive de données imposent au chercheur une rigueur scientifique.

% citer cultoromics ? approche pas forcément idiote
La méthode à l'oeuvre ne consiste pas à lancer des calculs exploratoires sur de grands ensembles de données, trouver des motifs de répétitions et d'en tirer les conclusions qui en découlent. Au contraire, elle se fonde sur des hypothèses et des savoirs historiques pour les questionner dans les textes. Les nouvelles méthodes que nous avons décrites fonctionnent en tandem avec la théorie littéraire et l'histoire littéraire. Mettre de côté les immenses savoirs hérités de l'analyse littéraire traditionnelle serait contre-productif et réduirait les capacités d'interprétation des études computationnelles. 

\section{La modélisation en études littéraires}

Dans la dernière décennie, le champ des humanités numériques a pu profiter des grands progrès des méthodes quantitatives. En effet, s'éloignant de la mesure de variables ou de fréquences d'occurrences dans les textes, les recherches actuelles utilisent désormais des modèles statistiques pour appréhender des concepts littéraires. Les recherches sont passées de méthodes statistiques descriptives à des méthodes prédictives de modélisation.

Dans ce mémoire, nous mettons en oeuvre ce genre de techniques et nous voulons ici préciser les modalités de leur emploi. L'introduction de l'apprentissage machine dans les processus de réflexion des sciences humaines va au delà des mesures en lecture distante. Selon Richard J. So, \enquote{La lecture à distance est innovante parce qu'elle a introduit non seulement des \enquote{données} ou des \enquote{algorithmes} dans les études littéraires, mais aussi, et surtout, la modélisation quantitative comme forme de raisonnement et d'analyse}\footcites{so_all_2017}. La modélisation statistique permet un changement de perspective car elle introduit dans les humanités numériques la notion d'incertitude et de probabilité de la mesure. Elle ne présente pas des résultats tranchés mais plutôt un compte rendu de ce que le modèle a cherché à mesurer et les limites de sa capacité à produire le résultat.

%Pour Ted Underwood, \enquote{Les modèles statistiques ne sont qu'une forme de preuve de plus, à évaluer avec toutes les autres. Cet article recommande les chiffres aux spécialistes de la littérature, non pas comme une forme de preuve unique et fiable, mais comme un langage descriptif flexible, particulièrement adapté aux questions historiques de perspective et de degré}\footcites{underwood_machine_2020}.

Pour comprendre plus précisément le changement de paradigme auquel nous avons à faire, nous pouvons encore une fois revenir à Franco Moretti et son article \enquote{Style, Inc : Réflexions sur 7 000 titres}\footcites{moretti_style_2009}. Le chercheur y atteste la baisse de la longueur des titres de romans dans le temps. Il fait des mesures simples, avec des moyennes par année de la longueur des titres, et les projette sur un graphe. Ce sont des analyses statistiques descriptives, qui ne permettent pas de conclure définitivement sur la portée de la tendance ou la significativité statistique des résultats. Andrew Piper porte l'idée que la modélisation littéraire introduit \enquote{une manière relativiste de penser la traversée des échelles d'analyse critique}\footcites{piper_think_2017}, c'est à dire que les modèles statistiques permettent une approche plus nuancée sur les données et les résultats quantifiés.
%Avant piper : Il serait aujourd'hui compliqué de faire sans une régression linéaire par exemple, pour attester et modéliser la tendance.
Dans le contexte des études littéraires computationnelles, l'utilisation récente de l'apprentissage automatique dans des tâches complexes de classification sur des textes littéraires a été l'objet de nombreuses recherches, telles que l'étude des spécificités intrinsèques d'un texte de fiction en comparaison à d'autres productions textuelles (philosophie, essais, ...). Le travail d'Andrew Piper\footcites{piper_fictionality_2016} montre que les indices textuels et leur modélisation permettent de reconnaître et d'attester un texte comme étant fictionnel ou non-fictionnel avec plus de 94\% d'efficacité. Si la tâche est en elle-même peu complexe, et que l'on pouvait s'attendre à obtenir de tels résultats, Andrew Piper explique que :
\begin{displayquote}
\enquote{l'intérêt du point de vue quantitatif est qu'il nous permet de mieux comprendre la manière dont un type d'écriture particulier signale aux lecteurs une orientation particulière [...]. Cela n'exclut pas les innombrables façons dont les lecteurs peuvent trouver leur propre version de l'importance du roman. Mais cela nous permet de mieux comprendre le roman en tant que catégorie sociale}. 
\end{displayquote}

Ainsi, l'algorithme d'apprentissage machine prend les caractéristiques associées à chaque texte et calcule le degré auquel elles les distinguent ce texte comme appartenant à la catégorie qui leur est attribuée. Pour autant, Ted Underwood précise que \enquote{l'ordinateur ne sait rien de l'histoire littéraire : il ne modélise que les éléments que nous lui fournissons. Le modèle ne peut mesurer aucune dimension universelle du langage ; il indique simplement si un texte donné ressemble plus aux exemples du groupe A ou du groupe B}\footcites{underwood_machine_2020}.

Des travaux similaires se sont focalisés sur la notion de genre littéraire\footcites{underwood_mapping_2013}. Il faut bien comprendre que le concept de genre n'est pas arrêté, et que la frontière est fine entre, par exemple, la \textit{science fiction} et la \textit{fantasy}. L'apprentissage machine ne cherche pas à décrire une vérité qui serait absolue, mais cherche plutôt à détecter les contours de concepts que l'humanité a forgé au cours des écritures et lectures des siècles passés. 

On pourrait citer bien d'autres travaux, modélisant différents concepts littéraires avec différentes approches computationnelles mais nous terminons cette section avec un élément, souligné par le chercheur Richard J. So\footcites{long_literary_2016}, qui nous semble de première importance. Il s'agit de l'interprétabilité des modèles statistiques. En effet, c'est un des enjeux majeurs de la recherche en humanités numériques si l'on veut démocratiser ces approches et réconcilier les méthodes quantitatives avec les études littéraires traditionnelles et plus largement les sciences humaines et sociales. La plupart des modèles statistiques mettent en place des coefficients pour réaliser leurs inférences. Récupérer ces coefficients peut approfondir notre compréhension des prises de décisions de ces algorithmes\footcites{ribeiro_why_2016} et ainsi amener de nouvelles formes de raisonnement pour traiter de questions propres aux sciences humaines et sociales. Nous proposons dans ce mémoire une synthèse entre une approche historique de l'analyse des textes littéraires avec une approche computationnelle. Cette dernière, décrite notamment par le Text Lab de Chicago\footcites{long_literary_2016}, permet une approche plus empirique par l'utilisation des grands nombres pour interagir avec les textes littéraires, mais un retour à une granularité fine est essentiel pour comprendre les inférences réalisées par nos modèles statistiques.


%ouvrir la boite 
\bigskip 

\section{État de l'art : Le canon littéraire au révélateur des méthodes quantitatives}

Le canon littéraire est une notion très discutée dans les études littéraires computationnelles. On peut, une fois encore, citer les travaux du laboratoire littéraire de l'université de Stanford, dont certains pamphlets abordent cette notion. 

Une première approche a été de décrire quantitativement les listes qui constituaient les différents canons. Le pamphlet 8, \enquote{Between Canon and Corpus : Six Perspectives on 20th- Century
Novels}\footnote{\cite{algee-hewitt_between_2015} Published in\cite{moretti_canonarchive_2017}} caractérise le canon littéraire et montre le caractère exclusif de tels ensembles. Une approche similaire a été reproduite récemment, avec un travail sur les syllabus des études hispaniques dans les universités étasuniennes\footcites{gonzalez_measuring_2021}. Les chercheurs ont étudié la diversité du canon avec des mesures d'entropie des populations canoniques dans le temps. D'autres études essaient de caractériser la notion de canon littéraire par la composition de ces ensembles, notamment dans leur construction naissante\footcites{baird_examining_2021}.

D'autres études sont allées au delà du canon littéraire et ont repris la construction binaire du champ littéraire de Pierre Bourdieu\footnote{\cite{bourdieu_les_1992}, p 176}, entre popularité et prestige. Marc Verboord\footcites{verboord_classification_2003} a classifié les auteurs selon leur position dans le champ littéraire, en utilisant les spécificités propres au prestige et à la popularité. Une étude du laboratoire littéraire de Stanford, le pamphlet 17\footnote{J.D. Porter \enquote{Popularity/Prestige}, Pamphlets of the Stanford Literary Lab 17 (2018), url: \url{https://litlab.stanford.edu/LiteraryLabPamphlet17.pdf}. Published in \cite{moretti_canonarchive_2017}}, a montré que ces axes semblaient pertinents pour cartographier l'espace littéraire et culturel.

La seconde approche très populaire pour appréhender le canon littéraire est de mesurer dans les textes mêmes, à l'aide des méthodes du traitement automatique des langues, des différences entre ouvrages canoniques et non-canoniques. 

Le pamphlet 11\footnote{Mark Algee-Hewitt et al. \enquote{Canon/Archive. Large-scale Dynamics in the Literay Field}, Pamphlets of the Stanford Literary Lab 11 (2016), url: \url{https://litlab.stanford.edu/LiteraryLabPamphlet11.pdf}. Published in \cite{moretti_canonarchive_2017}}, \textit{Canon/Archive. Large-scale Dynamics in the Literay Field}, du laboratoire littéraire de Stanford est très instructif à cet égard. L'article montre que la variété lexicale et la quantité d'information sont plus importants dans les oeuvres retenues par le canon littéraire que dans les autres ouvrages. 

Dans une synthèse sur l'utilisation de la théorie de l'information pour mesurer des éléments propres aux sciences humaines\footcites{liddle_could_2019}, Dallas Liddle explique que la redondance perçue par le lecteur humain n'est pas la même que celle calculée statistiquement, et qu'il ne faudrait pas porter de conclusions hâtives. Pourtant, il montre qu'il y a bien une pression informative dans la sélection de textes littéraires.

Ted Underwood consacre un article\footcites{underwood_longue_2016} à la classification automatique du prestige littéraire fondée sur des données textuelles. Il travaille sur de la poésie et définit le prestige littéraire comme étant la probabilité d'un texte à être examiné dans des revues littéraires spécialisées. La question principale qu'il se posait était la suivante : \enquote{Est-ce que la frontière sociale entre le goût d'une élite et le reste de la production littéraire est associée à des différences stylistiques reconnaissables ?} Avec des outils simples du TAL (des sacs de mots) et un algorithme prédictif, (une régression logistique), Ted Underwood obtient de bons résultats, de l'ordre de 75\% d'efficacité pour son modèle statistique. Il montre que le discours littéraire contenu dans le texte est en lien avec la réception du-dit texte, et que ce lien est statistiquement solide.

Dans le sillage de ces découvertes, de nombreuses recherches ont investi la question du prestige littéraire. Leur prisme d'approche se focalise sur le style des ouvrages consacrés, et sa démarcation potentielle par rapport aux autres. Ce sujet est très abordé aux Pays-Bas, on peut citer notamment le travail de Karina van Koolen, qui montre que le degré de littérarité perçue par les humains est quantifiable et modélisable\footcites{koolen_literary_2020}. Deux articles d'Andreas van Cranenburgh et al\footcites{van_cranenburgh_vector_2019}\footcites{van_cranenburgh_data-oriented_2017}, explorent cette littérarité perçue à l'aide de vecteurs de mots et parviennent à des résultats intéressants. Le prestige littéraire est ainsi associé à un style et à une esthétique textuelle particuliers. L'article analyse les mots, les thèmes et les vecteurs de documents qui sont associés à cette dernière. Il est néanmoins difficile de caractériser cette esthétique, et ce mémoire essaiera de poursuivre ce travail de clarification, avec une double approche entre \textit{distant} et \textit{close reading}.

Un des articles fondamentaux pour notre état de l'art est le tout récent travail de Judith Brottrager et al\footcites{brottrager_predicting_2021}, qui analyse et compare la relation entre le concept de canon fondé sur les contextes des oeuvres et les éléments textuels intrinsèques à ces dernières. Les résultats témoignent d'une absence de corrélation évidente entre les deux méthodes, avec cependant des éléments intéressants à plus petite échelle.

Ces recherches empiriques sur le prestige littéraire sont peu présentes en France, et peu d'expérimentations ont eu lieu sur des corpus d'ouvrages francophones. Il faudrait citer une étude sur la sélection successive d'ouvrages au prix Goncourt 2020\footcites{bernard_goncourt_2021}, qui n'obtient pas de résultats convaincants sur un corpus limité. 

Ainsi, ce mémoire s'inscrit dans un contexte de recherches dynamiques dans le monde anglo-saxon. Peu de travaux ont été réalisés sur des données francophones, nous ouvrons donc un champ de recherche qui peut être foisonnant. Un premier temps sera donc consacré à recueillir des méta-données pertinentes pour construire un canon littéraire français. Dans un deuxième temps, nous modéliserons les spécificités textuelles du canon littéraire à l'aide des méthodes de l'apprentissage machine et du traitement automatique des langues. Nous essaierons différentes approches pour améliorer le niveau de l'état de l'art qui se trouve aux alentours de 75\% avec les travaux de Ted Underwood déjà cités\footcites{underwood_longue_2016}.
