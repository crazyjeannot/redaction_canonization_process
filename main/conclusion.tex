\part*{Conclusion}
\addcontentsline{toc}{part}{Conclusion}
\markboth{Conclusion}{Conclusion} 

Ce travail propose une étude en profondeur de la notion de canon littéraire. Nous avons produit un ensemble de critères, fondés sur des constats historiques, pour caractériser le canon littéraire de la réception contemporaine. L'objectif de ce rapport est d'enrichir la vision traditionnelle qui envisage le  canon comme une construction arbitraire, politique, idéologique, ou relative au hasard. Nous nous sommes intéressés au contenu textuel des ouvrages pour ajouter à cette définition un aspect formel, d'ordre esthétique et interne aux œuvres qui pourrait expliquer la sélectivité du canon littéraire. À l'aide d'un grand corpus de romans, des méthodes quantitatives de l'apprentissage machine et du traitement automatique des langues, nous avons pu modéliser en lecture distante la notion de canon littéraire. Cette modélisation se fonde sur les dynamiques textuelles qui régissent les standards du jugement esthétique et la filtration institutionnelle qui en résulte.

Un des apports principaux de ce mémoire a été de montrer qu'il existait une esthétique canonique, et qu'un modèle statistique pouvait prédire la canonicité avec 75\% à 90\% d'efficacité. S'il a été difficile d'interpréter et de caractériser cette esthétique, ou de revenir à des enjeux littéraires plus concrets, nous avons pu établir les analyses suivantes : 

\begin{itemize}
    \item notre modèle détecte une usure de l'esthétique canonique dans les dernières décennies de notre histoire littéraire. On pourrait discuter de la pertinence de la notion de canon sur les décennies plus contemporaines, puisque la filtration temporelle par les différentes institutions n'a pas encore \textit{tamisé} les ouvrages.  
    \item malgré cette usure contemporaine, nous avons montré une grande stabilité du jugement esthétique sur près de deux siècles. Notre modèle a été capable de détecter des lignes structurantes bien plus stables que décrites par l'histoire littéraire. Cela pourrait s'expliquer par la seule prise en compte de la réception contemporaine. De plus amples recherches seraient nécessaires pour conclure avec certitude.
\end{itemize}

Par ailleurs, en distinguant le canon du non-canon et en décrivant ce qui rend le roman canonique singulier dans son écriture, nous avons tenté de comprendre la fonction sociale remplie par ces textes particuliers. Les éléments textuels mis à jour sont les témoins d'une manière de \textit{faire littérature}, manière sélectionnée et acceptée par la réception. Ils composent ce que l'on peut appeler un \textit{dialecte canonique}, qui est constitué d'une langue sélectionnée et autorisée dans l'enseignement, instituée comme modèle de style pour des générations d'étudiants.

Ce travail ouvre de nombreuses perspectives de recherches, d'une part pour pallier les limites de la présente étude, et d'autre part, pour préciser l'esthétique canonique détectée. Notre démarche voulait saisir quantitativement les variables linguistiques qui caractérisent le phénomène social du prestige. La tâche était d'autant plus complexe qu'elle consistait à prédire des évènements arrivés à réception, c'est-à-dire après écriture. C'est pour ces raisons que nous avons opté pour une approche simple en \textit{sac de mots}, avec un canon uniforme et des méta-données réduites. L'enjeu était de vérifier cette hypothèse de langue littéraire. Il faudrait effectuer des recherches complémentaires pour comprendre exactement ce qui se joue dans le et les canons littéraires.

Une approche future serait de récupérer des méta-données en diachronie, c'est-à-dire au fur et à mesure de la filtration de la réception. On pourrait également fragmenter notre vision du canon par acteurs du champ littéraire (éditions, manuels scolaires, prestige académique, revues littéraires, ...). Nous pourrions aussi prendre des caractéristiques textuelles plus complexes, avec des techniques algorithmiques au niveau de l'état de l'art en traitement automatique des langues, comme par exemple les vecteurs de mots ou de paragraphes, ou encore la modélisation de sujet. 

Les processus de décanonisation n'ont pas été abordés dans ce mémoire. Ils mériteraient un travail entier à eux seuls. Il faudrait identifier les ouvrages célébrés mais oubliés, qui sortent du canon au fil du temps, et étudier les raisons de cette marginalisation littéraire. 

Ainsi, notre recherche s'est inscrite dans une réflexion sur la littérature et son canon en tant qu'objet de définition d'une culture nationale. Nous avons pu mettre au jour grâce à notre démarche quantitative une norme esthétique véhiculée par le canon et entretenue par la succession des politiques culturelles. Notre diagnostic témoigne des conditions sociologiques, idéologiques et culturelles de la production de la valeur littéraire et de ses critères de sélection. Ce travail est un modeste exemple de ce que peuvent apporter les méthodes quantitatives à la recherche en sciences humaines et sociales. Les humanités numériques permettent d'ouvrir de nouvelles perspectives de compréhension du monde pour façonner des savoirs historiques inédits et comprendre les normes sociales et politiques sur lesquelles s'est construite notre société.






