\part*{Conclusion}
\addcontentsline{toc}{part}{Conclusion}
\markboth{Conclusion}{Conclusion} 

Ainsi, ce travail propose une étude en profondeur de la notion de canon littéraire.

Nous avons produit un ensemble de critères, fondés sur des constats historiques, pour caractériser le canon littéraire de la réception contemporaine. L'objectif de ce rapport est d'enrichir la vision traditionnelle qui envisage le  canon comme une construction arbitraire, politique, idéologique, ou relative au hasard. Nous nous sommes intéressés au contenu textuel des ouvrages pour ajouter à cette définition un aspect formel, d'ordre esthétique et interne aux oeuvres qui pourrait expliquer la sélectivité du canon littéraire. À l'aide d'un grand corpus de romans, des méthodes quantitatives de l'apprentissage machine et du traitement automatique des langues, nous avons pu modéliser en lecture distante la notion de canon littéraire. Cette modélisation se fonde sur les dynamiques textuelles qui régissent les standards du jugement esthétique et la filtration institutionnelle qui en résulte.

Un des apports principaux de ce mémoire a été de montrer qu'il existait une esthétique canonique, et qu'un modèle statistique pouvait prédire correctement la canonicité de 75\% à 90\% d'efficacité, selon l'échelle de cette dernière. S'il a été difficile d'interpréter et de caractériser cette esthétique, ou de revenir à des enjeux littéraires plus concrets, nous avons pu développer quelques résultats : notre modèle détecte une usure de l'esthétique canonique dans les dernières décennies de notre histoire littéraire. Plus qu'une usure, on pourrait discuter de la pertinence de la notion de canon sur les décennies plus contemporaines, puisque la filtration temporelle par les différentes institutions n'a pas encore tamisé le canon.  

Malgré cela, nous avons montré une grande stabilité du jugement esthétique sur près de deux siècles. Les évolutions des mouvements littéraires et le renouvellement générationnel des écrivains nous a amené à penser une réception sensible à ces changements historiques profonds, mais notre modèle a été capable de détecter des lignes structurantes bien plus stables que prévu par l'histoire littéraire. Cela peut s'expliquer par la prise en compte uniquement de la réception contemporaine, et de plus amples recherches seraient nécessaires pour conclure complètement.

En essayant de distinguer le canon du non-canon, en trouvant ce qui rend le roman canonique singulier dans son écriture, nous avons tenté de comprendre la fonction sociale remplie par ces textes particuliers. Les éléments textuels mis à jour sont les témoins d'une certaine façon de \textit{faire littérature}, sélectionnée et acceptée par la réception. Ils composent ce que l'on peut appeler un \textit{dialecte canonique}, qui est constitué d'une langue sélectionnée et autorisée dans l'enseignement, instituée comme modèle de style pour des générations d'étudiants.

Ce travail ouvre de nombreuses perspectives de recherches, d'une part pour pallier les limites de la présente étude, et d'autre part pour préciser l'esthétique canonique détectée. 

Notre démarche voulait détecter quantitativement les variables linguistiques qui caractérisent le phénomène social du prestige. La tâche était d'autant plus complexe qu'elle consistait en prédire des évènements arrivés à réception, c'est-à-dire après écriture. C'est pour ces raisons que nous avons opté pour une approche simple en sac de mot, avec un canon uniforme et des méta-données réduites. L'enjeux était de détecter cette hypothèse de langue littéraire, et il faudrait effectuer des recherches complémentaires pour comprendre exactement ce qui se joue dans le et les canons littéraires.

Une approche future pourrait être de récupérer des méta-données en diachronie, c'est-à-dire au fur et à mesure de la filtration de la réception. On pourra aussi fragmenter notre vision du canon, avec différents canons pour différents acteurs du champ littéraire (éditions, manuels scolaires, prestige académique, revues littéraires, ...). Nous pourrions prendre des caractéristiques textuelles plus complexes, avec des techniques algorithmiques au niveau de l'état de l'art en traitement automatique des langues, comme par exemple les vecteurs de mots ou de paragraphes, ou encore la modélisation de sujet. 

Des éléments que nous n'avons pas abordé dans ce mémoire mais qui mériteraient un travail entier à eux seuls sont les processus de décanonisation. Il faudrait identifier les ouvrages célébrés mais oubliés, qui sortent du canon au fil du temps, et étudier les raisons de cette marginalisation littéraire. 







