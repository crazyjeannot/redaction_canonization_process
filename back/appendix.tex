\appendix
\part*{Annexes}
\addcontentsline{toc}{part}{Annexes}
\pagestyle{myheadings}
\markboth{Annexes}{Annexes}

\section{Disponibilité du code}\label{code}

Illustrations, code et données du mémoire disponibles sur Github à cette adresse : \url{https://github.com/crazyjeannot/canonization_process}. 


\newpage
\section{Le canon des auteurs}\label{canon_auteur}

\begin{xltabular}{\textwidth}{lrr}
\toprule
                    Auteurs du canon &  Nombre de romans classiques &  Nombre total de romans \\
\midrule
                    Honore De Balzac &         19 &               85 \\
                          Emile Zola &         17 &               22 \\
                      Romain Rolland &         10 &               12 \\
                   Guy De Maupassant &         10 &               21 \\
                     Alexandre Dumas &          8 &               88 \\
                         Victor Hugo &          8 &               12 \\
      François René De Chateaubriand &          7 &                7 \\
                             Colette &          5 &               18 \\
                         Jules Verne &          4 &               28 \\
                     Alphonse Daudet &          4 &               21 \\
                         George Sand &          4 &               51 \\
                            Stendhal &          4 &                7 \\
                        Albert Camus &          4 &                4 \\
         Edmond Et Jules De Goncourt &          4 &                7 \\
            Jules Barbey D'Aurevilly &          3 &                8 \\
                        Louis Aragon &          3 &                7 \\
                     Patrick Modiano &          3 &               21 \\
                    Gustave Flaubert &          3 &                7 \\
                          Jean Giono &          3 &                6 \\
                     Michel Tournier &          3 &               11 \\
                         Romain Gary &          2 &                9 \\
                     Maurice Leblanc &          2 &               33 \\
                      Claude Farrère &          2 &                5 \\
                         Michel Deon &          2 &                3 \\
                       Georges Perec &          2 &               10 \\
                        Jean Echenoz &          2 &               16 \\
                  Simone De Beauvoir &          2 &                7 \\
                    Georges Bernanos &          2 &                8 \\
                   Nathalie Sarraute &          2 &                6 \\
                 Joris Karl Huysmans &          2 &               10 \\
            Antoine De Saint Exupery &          2 &                4 \\
                       André Malraux &          2 &                3 \\
                        Jules Renard &          2 &                4 \\
                       Pierre Michon &          2 &                6 \\
                        Sainte-Beuve &          2 &                2 \\
                   Delphine De Vigan &          2 &                4 \\
                    Marguerite Duras &          2 &                8 \\
                       Laurent Gaude &          2 &                4 \\
                Marguerite Yourcenar &          2 &                9 \\
                  Maylis De Kerangal &          2 &                5 \\
                          Andre Gide &          2 &               10 \\
                        Claude Simon &          2 &               11 \\
                       Joseph Kessel &          2 &                5 \\
                        Jules Vallès &          2 &                4 \\
                 Alain Robbe-Grillet &          2 &                6 \\
                      Henri Barbusse &          2 &                2 \\
                   Théophile Gautier &          2 &                6 \\
                       Mathias Enard &          1 &                4 \\
                     Raymond Queneau &          1 &                4 \\
                       Marcel Pagnol &          1 &                7 \\
                  Frederic Beigbeder &          1 &                5 \\
                    Antoine Volodine &          1 &                9 \\
                       Emile Moselly &          1 &                2 \\
               Mme Tarbe Des Sablons &          1 &                1 \\
                      Octave Mirbeau &          1 &                5 \\
                        Henri Murger &          1 &                1 \\
                     Patrick Deville &          1 &                1 \\
           Alphonse De Chateaubriant &          1 &                2 \\
                   Philippe Grimbert &          1 &                1 \\
               Jean-Christophe Rufin &          1 &                3 \\
                        Jules  Verne &          1 &                2 \\
                     Paul Fils Feval &          1 &                6 \\
        Jean Marie Gustave Le Clézio &          1 &                4 \\
                        Marie NDiaye &          1 &                1 \\
                    Félicien Marceau &          1 &                1 \\
                      Eugène Ionesco &          1 &                1 \\
                       Henry Bauchau &          1 &                4 \\
                      Henry Bordeaux &          1 &                3 \\
                          Paul Feval &          1 &               59 \\
                       Olivier Rolin &          1 &                8 \\
                      Alain Fournier &          1 &                1 \\
                           Leon Bloy &          1 &                3 \\
                    Raymond Radiguet &          1 &                3 \\
                    Pierre Combescot &          1 &                1 \\
                              Magali &          1 &                4 \\
                        Julien Gracq &          1 &                3 \\
              Louis Ferdinand Celine &          1 &                4 \\
                   Benjamin Constant &          1 &                1 \\
                   Georges Rodenbach &          1 &                2 \\
                    Philippe Claudel &          1 &                4 \\
                   Comtesse De Segur &          1 &               11 \\
                       Michel Leiris &          1 &                1 \\
                      Marie Cardinal &          1 &                1 \\
                     Cesarie Farrenc &          1 &                1 \\
                    Emmanuel Carrere &          1 &                6 \\
                        Erik Orsenna &          1 &                1 \\
                        Albert Cohen &          1 &                1 \\
                         Marc Dugain &          1 &                1 \\
                    Christian Gailly &          1 &               10 \\
             Jean Philippe Toussaint &          1 &                2 \\
                      Samuel Beckett &          1 &                4 \\
                        Pascal Laine &          1 &                1 \\
                               Delly &          1 &               95 \\
                  Marie Darrieussecq &          1 &                8 \\
           Charles Victor Arlincourt &          1 &                7 \\
                  Michel Houellebecq &          1 &                5 \\
                       Louis Pergaud &          1 &                3 \\
                    Eugene Fromentin &          1 &                1 \\
                      Sylvie Germain &          1 &                8 \\
                       Alice Zeniter &          1 &                1 \\
                         Pierre Loti &          1 &               31 \\
                     Jacques Laurent &          1 &                6 \\
                    Didier Daeninckx &          1 &               21 \\
                       Paul Verlaine &          1 &                4 \\
                        Roger Vercel &          1 &                1 \\
                     Georges Duhamel &          1 &                5 \\
                     Pascal Quignard &          1 &               12 \\
                        Marie Nimier &          1 &                1 \\
                       Marcel Proust &          1 &                7 \\
            Emile Ajar - Gary Romain &          1 &                4 \\
                       Marcel Schwob &          1 &                2 \\
                       Pierre Benoit &          1 &                2 \\
                    Maurice Genevoix &          1 &                2 \\
                          Jean Rolin &          1 &                4 \\
                    Alfred De Musset &          1 &                2 \\
                     Alfred De Vigny &          1 &                3 \\
                       Nina Bouraoui &          1 &                7 \\
                          Paul Nizan &          1 &                3 \\
                    Mathieu Riboulet &          1 &                4 \\
                    Francois Mauriac &          1 &                1 \\
                       Louise Michel &          1 &                1 \\
                Comte De Lautreamont &          1 &                1 \\
                   Marguerite Audoux &          1 &                6 \\
                    Catherine Cusset &          1 &                2 \\
                        Annie Ernaux &          1 &                8 \\
                        Alice Ferney &          1 &                6 \\
                   Chatrian Erckmann &          1 &               20 \\
                      Paule Constant &          1 &                1 \\
\bottomrule

\end{xltabular}


\newpage


\section{Le canon des œuvres}\label{canon_roman}
\begin{xltabular}{\textwidth}{lrr}
\toprule
Date &  Auteur &Titre \\
\midrule
1811 &       Chateaubriand-François-Rene-de &                     Oeuvres-completes \\
1816 &                    Constant-Benjamin &                               Adolphe \\
1821 &            Arlincourt-Charles-Victor &                          Le-Solitaire \\
1829 &                          Hugo-Victor &         Le-dernier-jour-d-un-condamne \\
1830 &                             Stendhal &                   Le-Rouge-et-le-noir \\
1831 &                          Hugo-Victor &                   Notre-Dame-de-Paris \\
1831 &                          Hugo-Victor &                   Notre-Dame-de-Paris \\
1832 &                      Vigny-Alfred-de &                                Stello \\
1834 &                         Sainte-Beuve &                               Volupte \\
1834 &                          Hugo-Victor &                          Claude-Gueux \\
1834 &                         Sainte-Beuve &                               Volupte \\
1835 &                    Gautier-Theophile &                Mademoiselle-de-Maupin \\
1836 &                     Musset-Alfred-de &   La-Confession-d-un-enfant-du-siecle \\
1839 &                             Stendhal &                La-Chartreuse-de-Parme \\
1840 &                          Sand-George &                               Pauline \\
1842 &                     Balzac-Honore-de &                La-Femme-de-trente-ans \\
1842 &                     Balzac-Honore-de &                               Beatrix \\
1842 &                     Balzac-Honore-de &                 Le-Contrat-de-mariage \\
1843 &                      Dumas-Alexandre &                          Le-Corricolo \\
1843 &                     Balzac-Honore-de &                      Le-Cure-de-Tours \\
1843 &                     Balzac-Honore-de &                        Le-Pere-Goriot \\
1843 &                     Balzac-Honore-de &                     Illusions-perdues \\
1843 &                     Balzac-Honore-de &                             Pierrette \\
1843 &                     Balzac-Honore-de &                       Eugenie-Grandet \\
1844 &                     Balzac-Honore-de &                    Le-Colonel-Chabert \\
1844 &                     Balzac-Honore-de & Splendeurs-et-miseres-des-courtisanes \\
1844 &       Chateaubriand-François-Rene-de &                          Vie-de-Rance \\
1844 &                     Balzac-Honore-de &                 Le-Lys-dans-la-vallee \\
1844 &                      Dumas-Alexandre &               Les-Trois-Mousquetaires \\
1845 &                      Dumas-Alexandre &              Le-Comte-de-Monte-Cristo \\
1845 &                     Balzac-Honore-de &                  Un-debut-dans-la-vie \\
1845 &                     Balzac-Honore-de &                                 Adieu \\
1845 &                      Dumas-Alexandre &                       Vingt-ans-apres \\
1845 &                      Dumas-Alexandre &                       La-Reine-Margot \\
1846 &                          Sand-George &                     La-Mare-au-Diable \\
1846 &                     Balzac-Honore-de &                Physiologie-du-mariage \\
1846 &                Mme-Tarbe-Des-Sablons &                              Isabelle \\
1846 &                     Balzac-Honore-de &                         Louis-Lambert \\
1846 &                     Balzac-Honore-de &                Une-tenebreuse-affaire \\
1846 &                     Balzac-Honore-de &                    Le-Cure-de-village \\
1846 &                      Dumas-Alexandre &                  La-Dame-de-Monsoreau \\
1846 &                     Balzac-Honore-de &                Le-Medecin-de-campagne \\
1848 &                      Dumas-Alexandre &                  La-dame-aux-camelias \\
1849 &       Chateaubriand-François-Rene-de &                Memoires-d-Outre-Tombe \\
1849 &       Chateaubriand-François-Rene-de &                Memoires-d-Outre-Tombe \\
1849 &       Chateaubriand-François-Rene-de &                Memoires-d-Outre-Tombe \\
1849 &                          Sand-George &                     La-petite-Fadette \\
1849 &       Chateaubriand-François-Rene-de &                Memoires-d-Outre-Tombe \\
1849 &       Chateaubriand-François-Rene-de &                Memoires-d-Outre-Tombe \\
1850 &                      Dumas-Alexandre &              Le-Vicomte-de-Bragelonne \\
1854 &             Barbey-d-Aurevilly-Jules &                          L-ensorcelee \\
1855 &                     Balzac-Honore-de &                           Les-Paysans \\
1855 &                             Stendhal &                 Chroniques-italiennes \\
1857 &                           Feval-Paul &                              Le-Bossu \\
1857 &                     Flaubert-Gustave &                         Madame-Bovary \\
1858 &                    Segur-comtesse-de &                Les-Malheurs-de-Sophie \\
1860 &                    Erckmann-Chatrian &                   Contes-fantastiques \\
1860 &          Goncourt-Edmond-et-Jules-de &                      Charles-Demailly \\
1862 &                          Hugo-Victor &                        Les-Miserables \\
1862 &                     Flaubert-Gustave &               Salammbô \\ 
1863 &                          Verne-Jules &               Cinq-Semaines-en-ballon \\
1863 &                     Fromentin-Eugene &                             Dominique \\
1863 &                      Farrenc-Cesarie &                           La-jalousie \\
1863 &                    Gautier-Theophile &                 Le-capitaine-Fracasse \\
1864 &                          Verne-Jules &          Voyage-au-centre-de-la-Terre \\
1864 &             Barbey-d-Aurevilly-Jules &              Le-Chevalier-des-Touches \\
1864 &          Goncourt-Edmond-et-Jules-de &                    Germinie-Lacerteux \\
1864 & Goncourt-Edmond-de-Goncourt-Jules-de &                        Renee-Mauperin \\
1866 &                          Hugo-Victor &            Les-travailleurs-de-la-mer \\
1868 &                           Zola-Emile &                        Therese-Raquin \\
1869 &                          Hugo-Victor &                       L'homme-qui-rit \\
1869 &                 Lautreamont-comte-de &                Les-chants-de-Maldoror \\
1869 &                         Murger-Henri &            Scenes-de-la-vie-de-Boheme \\
1869 &                     Flaubert-Gustave & L'education-sentimentale \\  
1870 &                           Zola-Emile &                 La-fortune-des-Rougon \\
1870 &                          Verne-Jules &      Vingt-mille-lieues-sous-les-mers \\
1872 &                           Zola-Emile &                              La-curee \\
1873 &                           Zola-Emile &                    Le-ventre-de-Paris \\
1873 &                      Daudet-Alphonse &                       Contes-du-lundi \\
1874 &                     Flaubert-Gustave &         La-tentation-de-saint-Antoine \\
1874 &                          Hugo-Victor &                    Quatrevingt-Treize \\
1874 &             Barbey-d-Aurevilly-Jules &                       Les-Diaboliques \\
1875 &                      Daudet-Alphonse &                                  Jack \\
1875 &                           Zola-Emile &             La-faute-de-l-abbe-Mouret \\
1876 &                           Zola-Emile &          Son-excellence-Eugene-Rougon \\
1876 &                          Sand-George &                                Horace \\
1877 &          Goncourt-Edmond-et-Jules-de &                   La-fille-Elisa \\
1877 &                     Flaubert-Gustave &                  Trois contes \\
1878 &                          Verne-Jules &            Un-capitaine-de-quinze-ans \\
1878 &                           Zola-Emile &                       Madeleine-Ferat \\
1879 &                  Huysmans-Joris-Karl &                     Les-soeurs-Vatard \\
1879 &                      Daudet-Alphonse &                 Lettres-de-mon-moulin \\
1880 &                           Zola-Emile &                                  Nana \\
1881 &                         Valles-Jules &                          Le-Bachelier \\
1881 &                         Valles-Jules &                              L-Enfant \\
1881 &                     Flaubert-Gustave &                   Bouvard-et-Pecuchet \\
1882 &                    Guy-de-Maupassant &                     Mademoiselle-Fifi \\
1883 &                           Zola-Emile &                  Au-bonheur-des-dames \\
1883 &                    Guy-de-Maupassant &                               Une-vie \\
1884 &                           Zola-Emile &                      La-joie-de-vivre \\
1884 &                    Guy-de-Maupassant &                         Clair-de-lune \\
1884 &                      Daudet-Alphonse &                                 Sapho \\
1884 &                  Huysmans-Joris-Karl &                             a-rebours \\
1884 &                    Guy-de-Maupassant &                             Au-soleil \\
1885 &                           Zola-Emile &                              Germinal \\
1885 &                          Verne-Jules &                       Mathias-Sandorf \\
1885 &                    Guy-de-Maupassant &                               Bel-ami \\
1885 &                    Guy-de-Maupassant &          Contes-du-jour-et-de-la-nuit \\
1886 &                        Michel-Louise &                              Memoires \\
1887 &                           Zola-Emile &                              La-terre \\
1887 &                            Bloy-Leon &                          Le-Desespere \\
1887 &                    Guy-de-Maupassant &                            Mont-Oriol \\
1887 &                          Loti-Pierre &                   Madame-Chrysantheme \\
1888 &                           Zola-Emile &                               Le-reve \\
1888 &                    Guy-de-Maupassant &                        Pierre-et-Jean \\
1889 &                    Guy-de-Maupassant &                        La-main-gauche \\
1890 &                    Guy-de-Maupassant &                           Notre-coeur \\
1890 &                           Zola-Emile &                       La-bete-humaine \\
1891 &                           Zola-Emile &                              L-argent \\
1892 &                           Zola-Emile &                            La-debacle \\
1892 &                    Rodenbach-Georges &                       Bruges-la-Morte \\
1893 &                           Zola-Emile &                     Le-docteur-Pascal \\
1893 &                        Verlaine-Paul &                           Mes-prisons \\
1893 &                         Renard-Jules &                          Coquecigrues \\
1893 &                      Leblanc-Maurice &                             Une-femme \\
1894 &                         Renard-Jules &                       Poil-de-carotte \\
1894 &                             Stendhal &                         Lucien-Leuwen \\
1895 &                      Leblanc-Maurice &                                Contes \\
1896 &                        Schwob-Marcel &                      Vies-imaginaires \\
1900 &                       Mirbeau-Octave &     Le-journal-d-une-femme-de-chambre \\
1904 &                       Rolland-Romain &                       Jean-Christophe \\
1904 &                       Rolland-Romain &                       Jean-Christophe \\
1904 &                       Rolland-Romain &                       Jean-Christophe \\
1905 &                       Rolland-Romain &                       Jean-Christophe \\
1905 &                       Farrere-Claude &                         Les-civilises \\
1907 &                        Moselly-Emile &                      Terres-lorraines \\
1908 &                       Rolland-Romain &                       Jean-Christophe \\
1908 &                       Barbusse-Henri &                               L-enfer \\
1908 &                       Rolland-Romain &                       Jean-Christophe \\
1908 &                       Rolland-Romain &                       Jean-Christophe \\
1909 &                       Farrere-Claude &                           La-Bataille \\
1910 &                    Audoux-Marguerite &                          Marie-Claire \\
1910 &                       Rolland-Romain &                       Jean-Christophe \\
1911 &                       Rolland-Romain &                       Jean-Christophe \\
1912 &                        Pergaud-Louis &                 La-Guerre-des-boutons \\
1912 &                       Rolland-Romain &                       Jean-Christophe \\
1913 &                       Bordeaux-Henry &                             La-Maison \\
1913 &                       Alain-Fournier &                     Le-grand-Meaulnes \\
1914 &                           Gide-Andre &                  Les-Caves-du-Vatican \\
1916 &                       Barbusse-Henri &                                Le-Feu \\
1919 &                        Benoit-Pierre &                           L-Atlantide \\
1922 &                         Aragon-Louis &            Les-aventures-de-Telemaque \\
1922 &                      Feval-Paul-fils &                              Le-Bossu \\
1923 &                     Radiguet-Raymond &                    Le-diable-au-corps \\
1923 &            Chateaubriant-Alphonse-de &                             La-Briere \\
1925 &                     Genevoix-Maurice &                              Raboliot \\
1925 &                        Proust-Marcel &                    Albertine-disparue \\
1926 &                     Bernanos-Georges &               Sous-le-soleil-de-Satan \\
1926 &                              Colette &                       La-Fin-de-Cheri \\
1926 &                        Kessel-Joseph &                           Les-Captifs \\
1927 &                              Colette &                       Le-Ble-en-herbe \\
1927 &                           Gide-Andre &                   Les-Faux-monnayeurs \\
1928 &                              Colette &                  La-naissance-du-jour \\
1929 &                     Bernanos-Georges &                               La-joie \\
1930 &                           Giono-Jean &                                Regain \\
1930 &                        Malraux-Andre &                        La-voie-royale \\
1930 &                              Colette &                                  Sido \\
1931 &             Saint-Exupery-Antoine-de &                           Vol-de-nuit \\
1932 &               Celine-Louis-Ferdinand &             Voyage-au-bout-de-la-nuit \\
1933 &                      Duhamel-Georges &                   Le-notaire-du-Havre \\
1933 &                        Malraux-Andre &                  La-Condition-Humaine \\
1934 &                         Vercel-Roger &                       Capitaine-Conan \\
1934 &                              Colette &               Les-Vrilles-de-la-vigne \\
1935 &                                Delly &                                Contes \\
1936 &                         Aragon-Louis &                   Les-Beaux-Quartiers \\
1936 &                           Giono-Jean &                  Les-vraies-richesses \\
1938 &                           Nizan-Paul &                       La-Conspiration \\
1939 &             Saint-Exupery-Antoine-de &                      Terre-des-hommes \\
1939 &                    Sarraute-Nathalie &                             Tropismes \\
1939 &                        Leiris-Michel &                         L-Age-d-homme \\
1942 &                         Camus-Albert &                            L-etranger \\
1944 &                         Camus-Albert &                      Le-premier-homme \\
1947 &                         Camus-Albert &                              La-peste \\
1950 &                     Duras-Marguerite &        Un-barrage-contre-le-Pacifique \\
1951 &                         Gracq-Julien &                  Le-rivage-des-Syrtes \\
1951 &                           Giono-Jean &                Le-Hussard-sur-le-toit \\
1951 &                       Beckett-Samuel &                                Molloy \\
1953 &                  Robbe-Grillet-Alain &                            Les-Gommes \\
1954 &                         Cohen-Albert &                   Le-livre-de-ma-mere \\
1954 &                   Beauvoir-Simone-de &                         Les-Mandarins \\
1954 &                   Beauvoir-Simone-de &                         Les-Mandarins \\
1956 &                          Gary-Romain &                   Les-Racines-du-Ciel \\
1958 &                        Kessel-Joseph &                               Le-lion \\
1959 &                      Queneau-Raymond &                   Zazie-Dans-Le-Metro \\
1960 &                        Pagnol-Marcel &                  Le-Temps-des-Secrets \\
1960 &                          Gary-Romain &                 La-promesse-de-l-aube \\
1960 &                         Simon-Claude &                 La-Route-des-Flandres \\
1965 &                        Perec-Georges &                            Les-choses \\
1967 &                      Tournier-Michel &   Vendredi-ou-les-limbes-du-Pacifique \\
1967 &                         Simon-Claude &                              Histoire \\
1968 &                 Yourcenar-Marguerite &                      L-Oeuvre-Au-Noir \\
1969 &                               Magali &                        La-prisonniere \\
1969 &                     Mauriac-Francois &             Un-adolescent-d-autrefois \\
1969 &                     Marceau-Felicien &                                Creezy \\
1970 &                          Deon-Michel &                   Les-poneys-sauvages \\
1970 &                      Tournier-Michel &                     Le-Roi-des-Aulnes \\
1971 &                      Laurent-Jacques &                           Les-betises \\
1971 &                      Tournier-Michel &            Vendredi-ou-la-vie-sauvage \\
1972 &                         Aragon-Louis &                              Aurelien \\
1972 &                      Modiano-Patrick &            Les-boulevards-de-ceinture \\
1973 &                          Deon-Michel &                         Un-Taxi-mauve \\
1973 &                       Ionesco-Eugene &                          Le-solitaire \\
1974 &                         Laine-Pascal &                        La-dentelliere \\
1974 &                 Yourcenar-Marguerite &                       Souvenirs-pieux \\
1975 &                       Cardinal-Marie &                 Les-mots-pour-le-dire \\
1975 &               Ajar-Emile-Gary-Romain &                     La-Vie-Devant-Soi \\
1977 &                      Modiano-Patrick &                     Livret-de-famille \\
1978 &                      Modiano-Patrick &            Rue-des-boutiques-obscures \\
1978 &                        Perec-Georges &                  La-Vie-mode-d-emploi \\
1980 &         Le-Clezio-Jean-Marie-Gustave &                                Desert \\
1983 &                    Sarraute-Nathalie &                               Enfance \\
1983 &                         Ernaux-Annie &                              La-place \\
1983 &                         Echenoz-Jean &                              Cherokee \\
1984 &                     Duras-Marguerite &                               L-Amant \\
1988 &                         Orsenna-Erik &                L-exposition-coloniale \\
1991 &                      Quignard-Pascal &              Tous-Les-Matins-Du-Monde \\
1991 &                        Michon-Pierre &                       Rimbaud-le-fils \\
1991 &                     Combescot-Pierre &                Les-filles-du-Calvaire \\
1994 &                         Camus-Albert &                      Le-premier-homme \\
1994 &                        Rolin-Olivier &                           Port-Soudan \\
1995 &                     Carrere-Emmanuel &                    La-classe-de-neige \\
1996 &                           Rolin-Jean &                        L-Organisation \\
1998 &                       Constant-Paule &            Confidence-pour-confidence \\
1998 &                          Dugain-Marc &              La-chambre-des-officiers \\
1998 &                     Daeninckx-Didier &                             Cannibale \\
1999 &                     Volodine-Antoine &                     Des-anges-mineurs \\
1999 &                         Echenoz-Jean &                          Je-m-en-vais \\
2001 &                  Robbe-Grillet-Alain &                            La-reprise \\
2001 &                        Gaude-Laurent &                                  Cris \\
2001 &                     Gailly-Christian &                       Un-soir-au-club \\
2001 &                Rufin-Jean-Christophe &                          Rouge-Bresil \\
2002 &                        Gaude-Laurent &                La-Mort-du-roi-Tsongor \\
2003 &                   Beigbeder-Frederic &                  Windows-on-the-World \\
2004 &                    Grimbert-Philippe &                             Un-secret \\
2005 &                       Germain-Sylvie &                                Magnus \\
2005 &                        Bouraoui-Nina &                 Mes-mauvaises-pensees \\
2006 &                         Nimier-Marie &                   La-Reine-du-silence \\
2007 &                    de-Vigan-Delphine &                             No-et-moi \\
2007 &                     Claudel-Philippe &                 Le-rapport-de-Brodeck \\
2008 &                        Enard-Mathias &                                  Zone \\
2008 &                        Bauchau-Henry &             Le-Boulevard-peripherique \\
2008 &                     Cusset-Catherine &                    Un-brillant-avenir \\
2009 &              Toussaint-Jean-Philippe &                   La-Verite-sur-Marie \\
2009 &                    Vigan-Delphine-de &               Les-heures-souterraines \\
2009 &                        Michon-Pierre &                              Les-Onze \\
2009 &                        N-Diaye-Marie &               Trois-femmes-puissantes \\
2010 &                   Kerangal-Maylis-de &                   Naissance-d-un-pont \\
2010 &                   Houellebecq-Michel &             La-Carte-et-le-territoire \\
2012 &                     Riboulet-Mathieu &            Les-oeuvres-de-misericorde \\
2012 &                      Deville-Patrick &                      Peste-et-cholera \\
2013 &                        Zeniter-Alice &                       Sombre-dimanche \\
2013 &                         Ferney-Alice &                     Cherchez-La-Femme \\
2013 &                   Darrieussecq-Marie &     Il-faut-beaucoup-aimer-les-hommes \\
2014 &                   Kerangal-Maylis-de &                   Reparer-les-vivants \\
\bottomrule
\end{xltabular}


\newpage
\section{Scores des motifs stylistiques}\label{perf_motif}

	\begin{table}[ht]
		\centering % Centre the table on the slide
		\begin{tabular}{l c c c c c}
			\toprule
    			 & precision & recall & f1-score & support & accuracy \\
			\toprule
			canon & 0.868 & 0.886 & 0.878 & 230 \\
			\midrule
			non-canon & 0.928 & 0.912 & 0.916 & 361 \\
			\midrule
			full dataset & & & & 592 & \textbf{0.902}\\
			\midrule
			macro-average & 0.896 & 0.898 & 0.896 & 591 \\
			\midrule
			weighted average & 0.902 & 0.898 & 0.902 & 591 \\

			\bottomrule
		\end{tabular}
	\caption{Résultats de l'évaluation du modèle en validation croisée pour les motifs}
	\end{table} 


\newpage 
\section{Passage remarquable dans \textit{L'éducation sentimentale}, de Gustave Flaubert}\label{eds}

Le souci de la vérité extérieure dénote la bassesse contemporaine ; et l’art deviendra, si l’on continue, je ne sais quelle rocambole au-dessous de la religion comme poésie, et de la politique comme intérêt. Vous n’arriverez pas à son but, – oui, son but ! – qui est de nous causer une exaltation impersonnelle, avec de petites œuvres, malgré toutes vos finasseries d’exécution. Voilà les tableaux de Bassolier, par exemple : c’est joli, coquet, propret, et pas lourd ! Ça peut se mettre dans la poche, se prendre en voyage ! Les notaires achètent ça vingt mille francs ; il y a pour trois sous d’idées ; mais, sans l’idée, rien de grand ! sans grandeur, pas de beau ! L’Olympe est une montagne ! Le plus crâne monument, ce sera toujours les Pyramides. Mieux vaut l’exubérance que le goût, le désert qu’un trottoir, et un sauvage qu’un coiffeur ! Frédéric, en écoutant ces choses, regardait Mme Arnoux. Elles tombaient dans son esprit comme des métaux dans une fournaise, s’ajoutaient à sa passion et faisaient de l’amour. Il était assis trois places au-dessous d’elle, sur le même côté. De temps à autre, elle se penchait un peu, en tournant la tête pour adresser quelques mots à sa petite fille ; et, comme elle souriait alors, une fossette se creusait dans sa joue, ce qui donnait à son visage un air de bonté plus délicate.Au moment des liqueurs, elle disparut. La conversation devint très libre ; M. Arnoux y brilla, et Frédéric fut étonné du cynisme de ces hommes. Cependant, leur préoccupation de la femme établissait entre eux et lui comme une égalité, qui le haussait dans sa propre estime. Rentré au salon, il prit, par contenance, un des albums traînant sur la table. Les grands artistes de l’époque l’avaient illustré de dessins, y avaient mis de la prose, des vers, ou simplement leurs signatures ; parmi les noms fameux, il s’en trouvait beaucoup d’inconnus, et les pensées curieuses n’apparaissaient que sous un débordement de sottises. Toutes contenaient un hommage plus ou moins direct à Mme Arnoux. Frédéric aurait eu peur d’écrire une ligne à côté. Elle alla chercher dans son boudoir le coffret à fermoirs d’argent qu’il avait remarqué sur la cheminée. C’était un cadeau de son mari, un ouvrage de la Renaissance. Les amis d’Arnoux le complimentèrent, sa femme le remerciait ; il fut pris d’attendrissement, et lui donna devant le monde un baiser. Ensuite, tous causèrent çà et là, par groupes ; le bonhomme Meinsius était avec Mme Arnoux, sur une bergère, près du feu ; elle se penchait vers son oreille, leurs têtes se touchaient ; et Frédéric aurait accepté d’être sourd, infirme et laid pour un nom illustre et des cheveux blancs, enfin pour avoir quelque chose qui l’intronisât dans une intimité pareille. Il se rongeait le cœur, furieux contre sa jeunesse. Mais elle vint dans l’angle du salon où il se tenait, lui demanda s’il connaissait quelques-uns des convives, s’il aimait la peinture, depuis combien de temps il étudiait à Paris. Chaque mot qui sortait de sa bouche semblait à Frédéric être une chose nouvelle, une dépendance exclusive de sa personne.

\newpage 
\section{Passage remarquable dans \textit{Un petit monde d'enfants}, de Elise Pressense}\label{enfants}

 — Maurice, tu oublies que je suis seule pour suffire à tout et que je suis moins forte qu’autrefois.  Il ne répondit pas et la mère continua :  —Oh ! mon pauvre enfant, si tu voulais m’écouter et te donner de la peine pour apprendre à lire et à écrire !... Tu es grand et fort ; l’année prochaine tu entrerais en apprentissage et tu pourrais bientôt m’aider. Pense comme tu serais heureux quand tu m apporterais l’argent que tu aurais gagné toi-même, et comme alors tout deviendrait facile.  Maurice balançait ses pieds sans répondre et tenait sa tête baissée pour ne pas rencontrer le regard de sa mère. Quelque chose lui disait qu’il serait en effet bien heureux de gagner de l’argent et de le lui donner, mais pour rien au monde il n’aurait voulu lui faire le plaisir d’en convenir. Il ne voulait pas la regarder non plus, car il sentait que peut-être il lui serait plus difficile de résister au muet appel de ses yeux qu’à ses paroles. Il se leva pour quitter la chambre.  — Ne vas-tu pas à l’école du dimanche ? demanda Madame Richard.  — Non, répondit-il en avançant un pied, puis l’autre. Je ne veux pas y aller avec ces souliers-là. Ça prend l’eau comme une éponge.  — Mais tu vas bien courir dans la neige.  Maurice ne répondit pas, et ferma la porte sur lui avec un bruit qui fit tressaillir tous les voisins dans leurs chambres.  Madame Richard, restée seule, pleura longtemps. Elle se sentait si faible, si seule, et l’égoïsme de son enfant lui faisait tant de mal ! Que n’avait-elle pas fait pour lui ! Que de nuits sans sommeil, que de journées de travail fiévreux ! Que de retranchements sur son strict nécessaire, à elle, pour donner à l’enfant un plaisir de plus ! Hélas ! elle le savait bien, elle l’avait gâté, et maintenant, elle recueillait les fruits amers de sa faiblesse.  Maurice descendit dans la cour, rendez-vous ordinaire de ses camarades. La neige tombée la veille avait été balayée et formait un amas grisâtre dans un coin. Les moineaux affamés sautillaient jusque sur le seuil de la porte, dans l’espoir d’y trouver quelques miettes de pain ; le soleil brillait, mais sans réchauffer, et les branches les plus déliées d’un arbre dont le sommet dépouillé apparaissait derrière un haut mur, se dessinaient avec une netteté parfaite contre le ciel, d’un bleu froid.  Maurice regarda les rangées d’innombrables fenêtres qui donnaient sur la cour. Tout était bien clos, personne ne songeait à mettre le nez dehors ; on n’entendait ni cris ni éclats de rire ; on eût dit que tous les habitants de la grande maison noire se livraient au repos du dimanche.  Cependant une petite figure, ensevelie dans un immense capuchon, parut sur le pas de la porte, et, toute frissonnante, s’y arrêta pour regarder les moineaux. Lydie avait un peu de pain dans sa poche ; elle le leur émietta, et les oisillons se mirent à voleter et à becqueter autour d’elle, tout ravis de ce festin inattendu.



\newpage 
\section{Projet ANR « Chapitres (XIXe-XXe siècles) »}\label{chapitre}

\bigskip

Je remercie toute l'équipe de l'ANR Chapitres et en particulier Aude Leblond pour m'avoir permis de réaliser mes recherches sur leur corpus. 

\bigskip

\textbf{Équipe de l'ANR Chapitres :}
\begin{itemize}
    \item Aude Leblond
    \item Claire Colin
    \item Thomas Conrad
    \item Marianne Reboul
    \item Raphaël Baroni
    \item Alexandre Gefen
    \item Camille Koskas
    \item Virginie Tahar
    \item Michel Bernard
    \item Alain Schaffner
    \item Jérémy Naïm
    \item Ugo Dionne
    \item Dimitri Garncarzyk
    \item Anaïs Goudmand
    \item Victoire Feuillebois
\end{itemize}

\bigskip

Lien : \url{https://chapitres.hypotheses.org/}.

\bigskip


Publication principale : \enquote{Pratique et poétiques du chapitre du XIXe au XXIe siècle}\footcites{colin_pratiques_2017}.