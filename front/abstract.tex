\section*{Résumé}
\addcontentsline{toc}{chapter}{Résumé}
Ce projet interroge le canon littéraire, une notion construite avec les biais de la société et modelée par les réceptions successives. L’objectif de ce rapport est de mettre en lumière l’existence de dynamiques textuelles qui assurent une longévité exceptionnelle à certains ouvrages et menacent au contraire la transmission d’une majorité d’autres. Les méthodes quantitatives du traitement automatique des langues et de l'apprentissage machine nous permettent de mettre au jour une esthétique intrinsèque au canon littéraire. Nous proposons une modélisation statistique de la canonicité avec des résultats prédictifs allant jusqu'à 90\% d'efficacité.


%, avec le contexte des œuvres,
% nous permettent de réaliser des analyses en \textit{lecture distante}\footnote{Franco Moretti. « Conjectures on World Literature », \textit{New Left Review}, 2000. republié dans \cite{moretti_distant_2013}} pour faire l’exégèse des tendances historiques qui caractérisent le canon littéraire par opposition avec les œuvres rejetées par ce dernier.


\medskip

\textbf{Mots-clés: études littéraires computationnelles ; canon littéraire ; classique ; littérature ; stylométrie ; histoire littéraire ; humanités numériques ; lecture distante ; traitement automatique de la langue ; apprentissage machine}

\textbf{Informations bibliographiques:} Jean Barré, \textit{Prédire la canonicité : Étude des dynamiques à grande échelle des
processus de canonisation}, mémoire de master 1 \og Humanités Numériques\fg{}, dir. [Thierry Poibeau, Jean-Baptiste Camps], Université Paris, Sciences \& Lettres, 2021.



\section*{Abstract}
\addcontentsline{toc}{chapter}{Abstract}
This project will interrogate the literary canon, a notion constructed with the biases of society and shaped by successive receptions. The aim of this report is to shed light on the existence of textual dynamics which ensure an exceptional longevity to some works and threaten the transmission of a majority of others. The quantitative methods of natural language processing and machine learning allow us to uncover an intrinsic aesthetic of the literary canon. We propose a statistical modeling of canonicity with predictive results of up to 90\% accuracy.

%, together with the context of the works,

% allow us to carry out analyses in \textit{distant reading}\footnote{Franco Moretti. "Conjectures on World Literature", \textit{New Left Review}, 2000. republished in \cite{moretti_distant_2013}} to exegete the historical trends that characterise the literary canon as opposed to the works rejected by it.

\medskip

\textbf{Keywords: computational literary studies ; canon ; classic ; archive ; literature ; stylometry ; digital humanities ; distant reading ; text mining ; natural langage processing ; machine learning}

\textbf{Bibliographic Information:} Jean Barré, \textit{Predicting canonicity, large scale dynamics of canonization processes}, M.A. thesis \og Digital Humanities\fg{}, dir. [Thierry Poibeau, Jean-Baptiste Camps], Université Paris, Sciences \& Lettres, 2021.


\clearpage